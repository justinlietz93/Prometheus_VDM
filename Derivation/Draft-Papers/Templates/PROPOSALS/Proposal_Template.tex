% ===========================================
% White Paper Proposal Template (≤5 pages)
% ===========================================
% Compiles on Overleaf with pdfLaTeX.
% Page limit is 5 total pages (figures + references included).
% This file provides: US Letter layout, float discipline, clean refs,
% and the exact section outline you specified.
%
% Usage:
%  - Fill in \ProposalTitle, \ProposalDate, authors/affiliations.
%  - Keep figures anchored near text; avoid float creep.
%  - Keep total length ≤5 pages; a compile-time warning is issued if exceeded.
%
\documentclass[11pt]{article}

% ---------- Page geometry & typography ----------
\usepackage[letterpaper,margin=1in]{geometry}
\usepackage{microtype}
\usepackage{parskip}            % block paragraphs (no indents)
\setlength{\parskip}{6pt}
\setlength{\parindent}{0pt}

% ---------- Math, units, and symbols ----------
\usepackage{amsmath,amssymb,mathtools}
\usepackage{siunitx}
\sisetup{detect-all}

% ---------- Figures and float discipline ----------
\usepackage{graphicx}
\usepackage{caption}
\usepackage{subcaption}
\usepackage[section]{placeins}  % \FloatBarrier at each section
\usepackage{float}              % allow [H] when truly necessary

% ---------- Lists and spacing ----------
\usepackage{enumitem}
\setlist{noitemsep,topsep=2pt}

% ---------- Hyperlinks and cross-references ----------
\usepackage[hidelinks]{hyperref}
\usepackage[nameinlink,capitalize]{cleveref}

% ---------- Headers/footers & page-count warning ----------
\usepackage{fancyhdr}
\usepackage{lastpage}
\usepackage{ifthen}
\pagestyle{fancy}
\fancyhf{}
\fancyhead[L]{White Paper Proposal}
\fancyhead[R]{\leftmark}
\fancyfoot[C]{Page \thepage\ of \pageref{LastPage} \, (limit: 5)}
\renewcommand{\headrulewidth}{0.4pt}

% ---------- Metadata (edit these) ----------
\newcommand{\ProposalTitle}{YOUR PROPOSAL TITLE}
\newcommand{\ProposalDate}{\today}

% For multiple authors/affiliations, use authblk; else keep simple:
\usepackage{authblk}
\author[1]{Justin K. Lietz}
\affil[1]{Neuroca, Inc.}
\title{\ProposalTitle}
\date{\ProposalDate}

% ---------- Optional: simple lemma/thm environments ----------
\usepackage{amsthm}
\theoremstyle{plain}
\newtheorem{lemma}{Lemma}
\theoremstyle{remark}
\newtheorem*{remark}{Remark}

% ---------- Hard page-limit warning at end of document ----------
\AtEndDocument{%
  \ifthenelse{\value{page}>5}{%
    \typeout{***********************************************}%
    \typeout{WARNING: Page limit exceeded (\thepage\ pages > 5).}%
    \typeout{***********************************************}%
  }{}%
}

\begin{document}
\maketitle

% =========================================================
% 1. Proposal Title and date
% =========================================================
\section{Proposal Title and date}
\textbf{Title:} \ProposalTitle

\textbf{Date:} \ProposalDate

% (Keep this section brief; the canonical title/date are also in the title block.)

% =========================================================
% 2. List of proposers and associated institutions/companies
% =========================================================
\section{List of proposers and associated institutions/companies}
\begin{itemize}
  \item \textbf{Justin K. Lietz} — Neuroca, Inc. (primary proposer)
  % \item Add additional proposers with affiliations as needed.
\end{itemize}

% =========================================================
% 3. Abstract  (<200 words)
% =========================================================
\section{Abstract}
% ≤200 words. State the motivation, the concrete goal(s), and what counts as success.
% Keep it decision-useful: what you will test, how you will know, and why it matters.
% (Abstract text goes here.)

% =========================================================
% 4. Background \& Scientific Rationale
% =========================================================
\section{Background \& Scientific Rationale}
% Audience: technically literate (physics/math/engineering) but not your subfield.
% Provide just enough to understand novelty, necessity, and plausibility.

\subsection*{Questions to consider (address concisely)}
\begin{itemize}
  \item How novel is this project?
  \item Why must this experiment be done?
  \item Are there target findings required for future work?
  \item What specific area of physics will this impact (e.g., quantum, gravity, EM)?
  \item What fundamental question/problem will it address?
  \item What criticisms might apply? If serious, how would you adapt?
\end{itemize}

\subsection*{Review criteria addressed}
\begin{enumerate}[label=\textbf{(\arabic*)}]
  \item Importance of the scientific questions addressed.
  \item Potential impact of the experiment.
  \item Clarity and reasonableness of the experimental approach.
  \item Planned level of rigor and discipline.
\end{enumerate}

% =========================================================
% 5. Experimental Plan
% =========================================================
\section{Experimental Plan}

\subsection{Experimental Setup and Diagnostics} % 5.1
% Required parameters; instrument list; diagnostics with counts; fabrication/software needs.
% Be specific and minimal. Use tables when helpful.

\subsubsection*{Known required parameters}
% e.g., domain sizes, sampling rates, control variables, seeds/commits if computational.

\subsubsection*{Diagnostics (with counts)}
% e.g., N oscilloscope channels, M cameras, K probes, loggers, scripts.

\subsubsection*{New equipment, tools, or scripts}
% Fabrication or coding that must be done before execution.

\subsection{Experimental runplan} % 5.2
% How 5.1 resources answer the scientific questions; runtime estimates; success/failure branches;
% publishing plan (figures, artifacts, and whitepaper formatting per PAPER_STANDARDS.md).

\paragraph{Resource employment}
% How each diagnostic/parameter supports each hypothesis or gate.

\paragraph{Estimated runtime}
% Wall-clock/compute time; parallelization plan if any.

\paragraph{Plan of action — success}
% What constitutes success; next steps if gates pass.

\paragraph{Plan of action — failure}
% Fallback routes; parameter sweeps; what falsifies the claim.

\paragraph{Results publication/display}
% Figures + paired CSV/JSON, seeds/commits; whitepaper format (see PAPER_STANDARDS.md).

% =========================================================
% 6. Personnel
% =========================================================
\section{Personnel}
% Roles and responsibilities. Keep concise and action-based.
\textbf{Justin K. Lietz}: concept, design, execution, analysis, and write-up.

% =========================================================
% 7. References  (counts toward 5-page limit)
% =========================================================
\section{References}
% Keep this terse and directly relevant. Examples below use manual bibitems
% to avoid external tooling; replace with your own entries.

\begin{thebibliography}{9}
\bibitem{Key1} A. Author, \emph{Title}, Journal \textbf{Volume}, pages (Year).
\bibitem{Key2} B. Author and C. Author, \emph{Title}, arXiv:xxxx.xxxxx.
\end{thebibliography}

% ---------- Keep figures near where they are cited ----------
% Example figure (delete if unused). Remember: figures count toward the 5-page limit.
% \begin{figure}[H]
%   \centering
%   \includegraphics[width=0.85\linewidth]{example-figure.png}
%   \caption{Descriptive caption with any paired artifact filename(s) noted.}
%   \label{fig:example}
% \end{figure}

\end{document}
