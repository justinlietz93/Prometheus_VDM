% =========================================================
% The Lietz Infinity Resolution Conjecture - arXiv style (VDM-aligned)
% Requires: arxiv.sty in the project
% =========================================================
\documentclass{article}

% ---- arXiv preprint look ----
\usepackage{arxiv}              % provided by the arXiv template
\usepackage[utf8]{inputenc}
\usepackage[T1]{fontenc}
\usepackage{lmodern}

% ---- math, figures, tables ----
\usepackage{amsmath, amssymb, amsthm, mathtools}
\usepackage{graphicx}
\usepackage{booktabs}
\usepackage{siunitx}
\sisetup{detect-all}
\usepackage{microtype}

% ---- refs & links (arXiv template uses natbib) ----
\usepackage{natbib}
\usepackage{doi}
\usepackage[hidelinks]{hyperref}

% ---- theorem environments ----
\newtheorem{conjecture}{Conjecture}
\newtheorem{corollary}{Corollary}
\newtheorem{definition}{Definition}
\newtheorem{proposition}{Proposition}

% ---- VDM gate + provenance helpers ----
\newenvironment{vdmgate}[2]{%
  \paragraph{Gate: #1} \emph{Threshold: #2.}%
  \par\noindent}{\medskip}
\newcommand{\provenance}[3]{\textbf{Commit:} \texttt{#1}\quad
  \textbf{Seed:} \texttt{#2}\quad
  \textbf{Artifacts:} \texttt{#3}}

% ---- Metadata ----
\title{The Lietz Infinity Resolution Conjecture:\\Tachyonic Hierarchy in Pulled-Front Systems}
\author{Justin K.\ Lietz\\
Neuroca, Inc.\\
\texttt{justin@neuroca.ai}}
\date{October 31, 2025}

% Short header
\renewcommand{\headeright}{A PREPRINT}
\renewcommand{\undertitle}{T8 Axiom Candidate — VDM Canon A8}
\renewcommand{\shorttitle}{Lietz Infinity Resolution}

\begin{document}
\maketitle

\begin{abstract}
We formalize a fundamental structural conjecture for metriplectic scalar-field systems with tachyonic origin ($V''(0)<0$) that admit pulled fronts with exponential tails: finite excess energy on unbounded domains necessitates organization into a finite-depth hierarchical partition with logarithmically-growing depth $N(L)=\Theta(\log(L/\lambda))$, scale-gap separation, and concentration of both energy and operational information at codimension-1 boundaries. This conjecture (designated VDM A8 candidate) provides a resolution mechanism for infinity via spontaneous emergence of hierarchical structure in field systems far from equilibrium. We present the precise mathematical framework, testable predictions via eight empirical gates (G1--G8), falsification criteria, and outline both analytical and numerical validation pathways. The conjecture bridges variational PDE theory, cosmological structure formation via spinodal decomposition, and information-theoretic measures of organized complexity.
\end{abstract}

\keywords{metriplectic dynamics \and tachyonic instability \and pulled fronts \and hierarchical structure \and spinodal decomposition \and energy concentration \and scaling laws}

% =========================================================
\section{Introduction}

\subsection{The Infinity Problem in Field Theory}

Physical field theories on unbounded domains face a fundamental tension: systems with local instabilities (negative curvature in the free energy) naturally spread structure, yet finite total energy demands some form of regularization. For metriplectic systems—those combining reversible (Hamiltonian) and irreversible (metric-dissipative) evolution—with tachyonic origin ($V''(0)<0$), this tension manifests acutely in the pulled-front regime where fronts propagate at the linear spreading speed $c_\star = 2\sqrt{Dr}$ with exponential tails.

\subsection{The Conjecture: Hierarchy as Infinity Resolution}

The \emph{Lietz Infinity Resolution Conjecture} posits that for such systems to maintain finite excess energy as domain size $L\to\infty$, they \emph{must} organize into a finite-depth hierarchical partition of codimension-1 interfaces with three key properties:
\begin{enumerate}
    \item \textbf{Logarithmic depth:} $N(L) = \Theta(\log(L/\lambda))$ where $\lambda$ is the exponential decay length;
    \item \textbf{Scale-gap separation:} successive levels exhibit geometric ratio $\rho\in(\rho_{\min},\rho_{\max})$;
    \item \textbf{Boundary concentration:} fractions $\alpha$ (energy) and $\alpha_\mathcal{I}$ (information) concentrate in thin boundary layers.
\end{enumerate}
This provides a \emph{spontaneous}, dynamically-emergent regularization mechanism distinct from externally-imposed cutoffs or compactification.

\subsection{Relation to VDM Canon}

This conjecture, if validated through the eight gates defined herein (Section~\ref{sec:gates}), will be designated as \textbf{Axiom A8} in the Variational Dynamics Model (VDM) canon. It serves as a foundational principle linking:
\begin{itemize}
    \item \textbf{Cosmology:} spinodal structure formation in tachyonic condensation scenarios;
    \item \textbf{Variational PDEs:} $\Gamma$-convergence and perimeter-energy concentration;
    \item \textbf{Information theory:} operational information localization at boundaries.
\end{itemize}

\subsection{Scope and Exclusions}

\textbf{Included:} Spatial dimensions $d\in\{1,2,3\}$; spinodal (tachyonic) onset; metriplectic structure; pulled-front regime; large-$L$ scaling.

\textbf{Excluded:} First-order barrier-crossing nucleation; exotic boundary forcing that externally imprints hierarchy; pushed-front regimes (where nonlinearity sets the front speed).

\subsection{Outline}

Section~\ref{sec:background} establishes formal definitions and mathematical setting. Section~\ref{sec:conjecture} states the conjecture precisely. Section~\ref{sec:predictions} details five operational predictions. Section~\ref{sec:methods} outlines analytical and numerical validation pathways. Section~\ref{sec:gates} defines the eight pass/fail gates and falsification criteria. Section~\ref{sec:discussion} interprets the conjecture's broader implications, and Section~\ref{sec:conclusions} summarizes next steps.

% =========================================================
\section{Background and Formal Setting}
\label{sec:background}

\subsection{State Space and Energy Functional}

\begin{definition}[State Space]
Let $\Omega\subset\mathbb{R}^d$ with $d\in\{1,2,3\}$ be a spatial domain, and $\phi:\Omega\to\mathbb{R}$ a scalar field in $H^1_{\text{loc}}(\Omega)$.
\end{definition}

\begin{definition}[Excess Energy Functional]
The excess energy relative to a stable state $\phi_\ast$ is
\begin{equation}
E_{\text{exc}}[\phi;\Omega] = \int_{\Omega}\Big(\kappa\,|\nabla \phi|^2 + V(\phi)-V(\phi_\ast)\Big)\,dx,
\end{equation}
where $\kappa>0$ is the gradient-energy coefficient and $V:\mathbb{R}\to\mathbb{R}$ is a $C^2$ potential with an unstable critical point at $\phi=0$ and at least one stable minimizer $\phi_\ast\neq 0$.
\end{definition}

\subsection{Tachyonic Origin and Metriplectic Evolution}

\begin{definition}[Tachyonic Origin]
The system has \emph{tachyonic origin} if $V''(0)<0$. Define $r\equiv -V''(0)>0$ as the tachyonic instability rate.
\end{definition}

This negative curvature at the quiet field $\phi=0$ drives spinodal decomposition—a globally unstable roll rather than barrier-crossing nucleation.

\begin{definition}[Metriplectic Evolution]
The field evolves via
\begin{equation}
\partial_t \phi = \underbrace{J(\phi)\,\frac{\delta \mathcal{H}}{\delta \phi}}_{\text{reversible}} + \underbrace{M(\phi)\,\frac{\delta \Sigma}{\delta \phi}}_{\text{dissipative}},
\end{equation}
where $J(\cdot)$ is antisymmetric (Hamiltonian bracket), $M(\cdot)$ is symmetric positive semidefinite (metric bracket), $\mathcal{H}$ is an energy-like functional, and $\Sigma$ is an entropy-like functional. This encodes causality and arrow-of-time without sacrificing time-reversal at the reversible level.
\end{definition}

For reaction--diffusion realizations, this reduces to
\begin{equation}
\partial_t \phi = D\nabla^2\phi + r\phi + f(\phi),
\label{eq:rd_form}
\end{equation}
with $D=\kappa/m$ (diffusion coefficient) and $f(\phi)$ nonlinear terms ensuring bounded solutions.

\subsection{Pulled-Front Regime}

\begin{definition}[Pulled Front]
A traveling interface connecting $\phi=0$ to $\phi=\phi_\ast$ is \emph{pulled} if its speed equals the linear spreading speed:
\begin{equation}
c_\star = 2\sqrt{Dr},
\label{eq:c_star}
\end{equation}
and the leading edge exhibits exponential decay $\phi(x)\sim A\,e^{-x/\lambda}$ with decay length $\lambda = \sqrt{D/r}$.
\end{definition}

Pulled fronts are selection mechanisms where the asymptotic tail, not the nonlinear core, controls the speed~\cite{vanSaarloos2003}. This contrasts with \emph{pushed} fronts where nonlinearity sets $c>c_\star$.

\subsection{Finite-Energy Admissibility}

\begin{definition}[Finite Excess Energy]
A family $\{\phi_L\}$ of states on domains $\Omega_L$ (e.g., cubes of side $L$) is \emph{finite-excess-energy} if
\begin{equation}
\sup_{L} E_{\text{exc}}[\phi_L;\Omega_L] < \infty \quad\text{as } L\to\infty.
\end{equation}
\end{definition}

This condition is non-trivial: naive homogeneous spreading would yield $E_{\text{exc}}\sim L^d$, violating finite-energy.

\subsection{Hierarchical Scale Breaks}

\begin{definition}[Hierarchical Partition]
A finite-depth hierarchical partition $\mathcal{P}=\{\Gamma_\ell\}_{\ell=1}^{N}$ of $\Omega$ is a nested sequence of codimension-1 interfaces satisfying:
\begin{enumerate}
    \item \textbf{Gap condition:} $\exists \rho\in(0,1), C\ge 1$ such that
    \begin{equation}
    \text{diam}(\Gamma_{\ell+1}) \in [\rho/C, C\rho]\cdot \text{diam}(\Gamma_{\ell}) \quad \forall\,\ell;
    \end{equation}
    \item \textbf{Finite depth:} $N<\infty$ for finite $L$, with $N(L)=\mathcal{O}(\log(L/\lambda))$ as $L\to\infty$;
    \item \textbf{Boundary concentration:} $\exists\alpha\in(0,1), \epsilon_0>0$ such that $\forall\epsilon<\epsilon_0$,
    \begin{equation}
    \liminf_{L\to\infty} \frac{ \int_{\mathcal{N}_\epsilon(\cup_\ell \Gamma_\ell)} \kappa|\nabla \phi_L|^2 \,dx }{ E_{\text{exc}}[\phi_L;\Omega_L] } \ge \alpha,
    \end{equation}
    where $\mathcal{N}_\epsilon(\cdot)$ is the $\epsilon$-tubular neighborhood.
\end{enumerate}
\end{definition}

Interpretation: energy concentrates in thin boundary layers (interpretation as codimension-1 perimeter energy), with a self-similar hierarchy bridging the microscale $\lambda$ and macroscale $L$.

\subsection{Information Density}

\begin{definition}[Operational Information Density]
Define a field-local information proxy $\mathcal{I}(x)$ such as
\begin{equation}
\mathcal{I}(x) = \log\!\big(1 + \tfrac{|\nabla \phi(x)|^2}{\sigma^2}\big)
\quad\text{or}\quad
\mathcal{I}(x) = \tfrac{1}{2}\log\!\det\!\big(I + \tau \,\nabla \phi(x)\nabla \phi(x)^\top\big),
\end{equation}
and require analogous boundary concentration: fraction $\alpha_\mathcal{I}$ in $\mathcal{N}_\epsilon(\cup_\ell \Gamma_\ell)$ as $L\to\infty$.
\end{definition}

This captures the notion that organized structure—where decisions/control/information-processing could occur—localizes at boundaries.

% =========================================================
\section{The Conjecture: Precise Statement}
\label{sec:conjecture}

\begin{conjecture}[Lietz Infinity Resolution — Tachyonic Hierarchy]
\label{conj:main}
For metriplectic scalar-field systems with tachyonic origin $V''(0)<0$ that admit pulled fronts with exponential tails and for which a family of states $\{\phi_L\}$ on $\Omega_L$ has finite excess energy in the limit $L\to\infty$, there must exist a hierarchical partition $\mathcal{P}=\{\Gamma_\ell\}_{\ell=1}^{N(L)}$ satisfying:
\begin{enumerate}
    \item \textbf{Gap condition:} scale separation with ratio $\rho\in(\rho_{\min},\rho_{\max})$;
    \item \textbf{Finite depth:} $N(L)=\mathcal{O}(\log(L/\lambda))$;
    \item \textbf{Boundary energy concentration:} fraction $\alpha>0$ in $\mathcal{N}_\epsilon(\cup_\ell\Gamma_\ell)$.
\end{enumerate}
\end{conjecture}

\textbf{Contrapositive form:} Absence of such hierarchical scale breaks implies either:
\begin{enumerate}
    \item[(a)] $\limsup_{L\to\infty} E_{\text{exc}}[\phi_L;\Omega_L] = \infty$ (energy blow-up), or
    \item[(b)] violation of the pulled-front bound $c\neq c_\star=2\sqrt{Dr}$.
\end{enumerate}

\begin{corollary}[Scaling Predictions]
\label{cor:scaling}
For large $L$:
\begin{itemize}
    \item \textbf{Depth:} $N(L)=\Theta\!\big(\log(L/\lambda)\big)$;
    \item \textbf{Energy scaling:} $E_{\text{exc}}(L)=\Theta(L^{d-1})$ (boundary-law) rather than $\Theta(L^d)$ (bulk);
    \item \textbf{Information concentration:} $\alpha_\mathcal{I}$ bounded away from 0.
\end{itemize}
\end{corollary}

\subsection{Interpretation}

This conjecture asserts that \emph{infinity is resolved via spontaneous hierarchy}: the interplay of tachyonic instability (driving spreading) and finite-energy constraint (resisting unbounded growth) forces the emergence of nested boundaries that concentrate both dynamical activity and information. The logarithmic depth arises from the exponential tail $\lambda$: to bridge scales from $\lambda$ to $L$ requires $\log(L/\lambda)$ doublings.

% =========================================================
\section{Operational Predictions}
\label{sec:predictions}

We detail five preregistered predictions (P1--P5) against which empirical tests will be benchmarked.

\subsection{P1: Depth vs.\ Size}

\begin{equation}
\frac{N(L)}{\log(L/\lambda)} \to c_N \in (0,\infty) \quad\text{as}\quad L\to\infty,
\end{equation}
with $c_N$ a universal constant (may depend on $d$ and potential class). Operationally: measure $N(L)$ for $L\in\{L_1,\dots,L_5\}$ spanning at least two decades; fit $N(L) = a + b\log(L/\lambda)$; verify $b\in[0.85, 1.15]$ (±15\% tolerance).

\subsection{P2: Boundary Energy Law}

\begin{equation}
\frac{E_{\text{exc}}(L)}{L^{d-1}} \to c_E \in (0,\infty) \quad\text{under unconstrained evolution}.
\end{equation}
Ablations suppressing hierarchy (e.g., penalizing interface count) should show trend toward $E_{\text{exc}}(L)\sim L^{d}$. Operationally: log--log fit $E_{\text{exc}}(L) = A\,L^\beta$; verify $\beta\in[d-1-0.1, d-1+0.1]$ (±0.1 tolerance).

\subsection{P3: Tail-Locked Scales}

Inter-level ratio $\rho\equiv \text{diam}(\Gamma_{\ell+1})/\text{diam}(\Gamma_{\ell})$ remains in $(\rho_{\min},\rho_{\max})$ with $\rho_{\max}/\rho_{\min}\le 2$ across $L$ and random seeds. Operationally: compute $\rho_\ell$ for each pair, verify $\rho\in[0.4,0.6]$ (golden-ratio vicinity expected from self-similarity arguments) with coefficient of variation $<0.15$.

\subsection{P4: Boundary Information Dominance}

\begin{equation}
\alpha, \alpha_\mathcal{I} \ge 0.6 \quad\text{(preregistered threshold)}.
\end{equation}
Operationally: for $\epsilon=\lambda/2$, compute
\begin{equation}
\alpha = \frac{ \int_{\mathcal{N}_\epsilon} \kappa|\nabla\phi|^2\,dx }{ E_{\text{exc}} }, \quad
\alpha_\mathcal{I} = \frac{ \int_{\mathcal{N}_\epsilon} \mathcal{I}(x)\,dx }{ \int_{\Omega}\mathcal{I}(x)\,dx },
\end{equation}
across at least three distinct boundary-detection masks; verify all $\ge 0.6$ with FDR $q\le 0.10$.

\subsection{P5: Pulled-Front Integrity}

Measured front speeds $c_{\text{meas}}$ satisfy
\begin{equation}
\Big|\frac{c_{\text{meas}}}{c_\star} - 1\Big| \le 0.02 \quad\text{when hierarchy allowed};
\end{equation}
deviate $>0.05$ when hierarchy penalized. Operationally: track front positions via level sets $\phi=\phi_\ast/2$; fit linear regression; compare to theoretical $c_\star=2\sqrt{Dr}$.

% =========================================================
\section{Methods: Analytical and Numerical Pathways}
\label{sec:methods}

\subsection{Analytical Track}

\subsubsection{1D Lower Bound (Gate G1)}

\textbf{Goal:} Prove that for tachyonic $V$ with pulled-front tail on $[0,L]$, any finite-energy connecting map $\phi:[0,L]\to[\phi(0),\phi_\ast]$ requires $N(L)\ge c\log(L/\lambda)$ interfaces (or multi-scale partition equivalent).

\textbf{Strategy:} Use energy lower bounds: each interface contributes $\sim\sigma$ (surface tension), and the exponential tail prevents concentration below scale $\lambda$. For $L\gg\lambda$, bridging scales demands $\mathcal{O}(\log(L/\lambda))$ doublings.

\subsubsection{$\Gamma$-Convergence Reduction (Gate G2)}

\textbf{Goal:} Show a $\Gamma$-convergence or perimeter-law reduction: interface energy concentrates on codimension-1 sets with surface tension $\sigma(V,\kappa)$, establishing $E_{\text{exc}}(L)\ge c\,\sigma\,L^{d-1}$.

\textbf{Strategy:} Follow Modica--Mortola pattern~\cite{ModicaMortola1977}: in small-interface-width limit, energy localizes to perimeter measure $\mathcal{H}^{d-1}$. Tie tail length $\lambda$ to scale gaps via exponential corrections.

\subsection{Numerical Track}

\subsubsection{Reaction--Diffusion Implementation}

Use canonical RD form:
\begin{equation}
\partial_t \phi = D\nabla^2\phi + r\phi - u\phi^2 - \lambda_3 \phi^3,
\label{eq:rd_numerical}
\end{equation}
with symmetric $\phi^4$ ($\lambda_3=0$) or biased $\phi^3+\phi^4$ (small $\lambda_3$).

\textbf{Measured quantities:}
\begin{itemize}
    \item $E_{\text{exc}}$ via numerical quadrature;
    \item $N(L)$ via interface detection (gradient thresholding + clustering);
    \item $\rho$ from successive level diameters;
    \item $\alpha$, $\alpha_\mathcal{I}$ via tubular neighborhoods;
    \item $c/c_\star$ from front tracking.
\end{itemize}

\subsubsection{Ablations: Interface Penalty}

Introduce regularizer $+\mu\,(\text{\#interfaces})$ or curvature penalty suppressing subdivision. Expect either energy blow-up or front-speed deviation per Conjecture~\ref{conj:main}.

\subsubsection{Echo-Steering Option}

Leverage VDM's CEG (Causal Entropy Gate) logic: metriplectic micro-nudges that accelerate hierarchy formation while preserving $c_\star$. This tests whether agency-aligned steering respects the conjecture's constraints.

\subsection{Variables and Units}

\begin{table}[t]
  \centering
  \caption{Primary variables and units.}
  \label{tab:variables}
  \begin{tabular}{@{}llll@{}}
  \toprule
  Variable & Description & Units & Range \\
  \midrule
  $\phi$ & Order parameter & dimensionless & $[0,1]$ \\
  $D$ & Diffusion coefficient & $\text{length}^2/\text{time}$ & $10^{-3}$--$10^{-1}$ \\
  $r$ & Tachyonic rate & $1/\text{time}$ & $0.1$--$10$ \\
  $\lambda$ & Decay length & length & $\sqrt{D/r}$ \\
  $L$ & Domain size & length & $10\lambda$--$1000\lambda$ \\
  $N$ & Hierarchy depth & count & $1$--$\log_2(L/\lambda)$ \\
  \bottomrule
  \end{tabular}
\end{table}

All simulations employ dimensionless units with $D=r=1$ (adjusting $L$ and $\lambda$ accordingly).

\subsection{Equipment and Software Stack}

\textbf{Hardware:} AMD Radeon (ROCm 5.x); 64 GB RAM; storage on NVMe SSD.

\textbf{Software:} Python 3.10+; NumPy 1.24+; SciPy 1.10+; PyTorch 2.0+ (GPU-accelerated PDE solver); Matplotlib/Seaborn (visualization).

\textbf{Precision:} Float64 (double precision) throughout to avoid spurious interface merging.

\subsection{Procedure}

\begin{enumerate}
    \item Initialize $\phi$ with small random perturbation around $\phi=0$ (seed-controlled).
    \item Evolve Eq.~\eqref{eq:rd_numerical} with periodic or absorbing BCs until quasi-steady (front propagation stabilized).
    \item Measure $E_{\text{exc}}$, $N$, $\rho$, $\alpha$, $\alpha_\mathcal{I}$, $c$.
    \item Repeat for $\ge 12$ random seeds per $(L,D,r)$ configuration.
    \item Perform ablation runs with interface penalty.
    \item Aggregate statistics; fit scaling laws; assess gate pass/fail.
\end{enumerate}

\subsection{Provenance}

\provenance{\texttt{<commit-hash>}}{\texttt{<seed>}}{\texttt{https://.../axioms/a8\_infinity\_resolution/}}

All code, manifests, and artifacts archived under VDM reproducibility standards. See \texttt{Derivation/code/outputs/axioms/a8\_infinity\_resolution/} for logs, figures, and reports.

% =========================================================
\section{Gates and Falsification Criteria}
\label{sec:gates}

We define eight pass/fail gates (G1--G8). \textbf{PASS} requires G1--G5 + at least one of G6--G7 + G8. \textbf{FAIL} if any of G1--G5 or G8 fail.

\begin{vdmgate}{G1: Theory-1D}{Lower bound $N(L)\ge c\log(L/\lambda)$ proven in 1D}
\textbf{Criterion:} Rigorous proof (or strong heuristic bound) that any finite-energy connecting map on $[0,L]$ with exponential tail requires $N(L)\ge c\log(L/\lambda)$ interfaces or equivalent multi-scale partition.

\textbf{Status:} Pending analytical work (Section~\ref{sec:methods}).

\textbf{Artifacts:} Theory note PDF + repository anchors.
\end{vdmgate}

\begin{vdmgate}{G2: Theory-$\Gamma$-style}{$\Gamma$-convergence or perimeter-law reduction}
\textbf{Criterion:} Show that interface energy concentrates on codimension-1 sets with surface tension $\sigma(V,\kappa)$, establishing $E_{\text{exc}}(L)\ge c\,\sigma\,L^{d-1}$.

\textbf{Status:} Pending analytical work.

\textbf{Artifacts:} Theory note PDF.
\end{vdmgate}

\begin{vdmgate}{G3: Numerics-Scaling}{Scaling law validation in 2D/3D}
\textbf{Criterion:} In 2D/3D RD sims, measure $E_{\text{exc}}(L)$ vs $L$ for $\ge 5$ points, $\times 12$ seeds. Log--log fit shows:
\begin{itemize}
    \item Slope $\beta\in[d-1-0.1, d-1+0.1]$;
    \item $N(L)\sim \log(L/\lambda)$ with slope $\in[0.85,1.15]$ (±15\%).
\end{itemize}

\textbf{Status:} Preregistered experiment suite.

\textbf{Artifacts:} \texttt{scaling\_E\_vs\_L.csv/json}, \texttt{scaling\_N\_vs\_L.csv/json}, figures.
\end{vdmgate}

\begin{vdmgate}{G4: Concentration}{Boundary energy and information fractions}
\textbf{Criterion:} $\alpha\ge 0.6$ and $\alpha_\mathcal{I}\ge 0.6$ across $\ge 3$ masks; FDR $q\le 0.10$.

\textbf{Status:} Preregistered.

\textbf{Artifacts:} \texttt{concentration\_alpha.csv/json}, \texttt{concentration\_alphaI.csv/json}.
\end{vdmgate}

\begin{vdmgate}{G5: Ablation}{Hierarchy suppression test}
\textbf{Criterion:} Penalizing interface count shows either:
\begin{itemize}
    \item $E_{\text{exc}}(L)$ trend toward $L^d$ (bulk scaling), or
    \item Front speed deviation $|c_{\text{meas}}/c_\star - 1|>0.05$.
\end{itemize}

\textbf{Status:} Preregistered ablation suite.

\textbf{Artifacts:} \texttt{ablation\_E\_vs\_L.csv/json}, \texttt{ablation\_front\_speed.csv/json}.
\end{vdmgate}

\begin{vdmgate}{G6: Robustness}{Cross-potential and cross-BC validation}
\textbf{Criterion:} Results hold across:
\begin{itemize}
    \item Potentials: $\phi^4$ symmetric and mildly biased $\phi^3+\phi^4$;
    \item Boundary conditions: periodic and absorbing;
    \item Mesh refinement: $2\times$ resolution doubling.
\end{itemize}

\textbf{Status:} Preregistered.

\textbf{Artifacts:} Aggregate robustness table.
\end{vdmgate}

\begin{vdmgate}{G7: Cross-Code}{Independent implementation}
\textbf{Criterion:} Independent implementation reproduces G3--G5 within stated error bars.

\textbf{Status:} Pending external collaboration.

\textbf{Artifacts:} Cross-code validation report.
\end{vdmgate}

\begin{vdmgate}{G8: Documentation}{VDM reproducibility compliance}
\textbf{Criterion:} All preregistration specs, code, manifests, and artifacts pass VDM reproducibility checks (hashes, provenance, logbooks).

\textbf{Status:} To be verified upon experiment completion.

\textbf{Artifacts:} Manifest JSONs, commit logs, artifact archives.
\end{vdmgate}

\subsection{Falsification Criteria}

\textbf{F1.} Existence of finite-excess-energy, large-$L$ states with no hierarchical partition (no gap, no boundary concentration) and fronts still at $c_\star$.

\textbf{F2.} Robust demonstrations (across seeds and domains) that $E_{\text{exc}}(L)=o(L^{d-1})$ or remains $O(1)$ without hierarchical boundaries.

\textbf{F3.} Empirical boundary-energy fraction $<0.3$ with stable pulled fronts.

Any of F1--F3 constitutes a strong falsification, prompting revision or withdrawal of the conjecture.

% =========================================================
\section{Discussion}
\label{sec:discussion}

\subsection{Physical Interpretation}

The Lietz Infinity Resolution Conjecture provides a \emph{mechanism} for how field theories with tachyonic instabilities avoid unphysical energy divergence: spontaneous hierarchical organization. This is not a passive boundary condition or external cutoff; it is an emergent structure driven by the interplay of spreading (tachyonic) and confinement (finite-energy).

\subsection{Connection to Cosmological Structure}

In VDM cosmogenesis, the universe undergoes spinodal decomposition following a tachyonic phase transition. The conjecture predicts that such a scenario naturally generates hierarchical structure (cf.\ cosmic web: voids, filaments, sheets, nodes) with characteristic scales set by the exponential correlation length $\lambda$. The boundary-energy concentration mirrors how mass and energy localize in walls and filaments.

\subsection{Information-Theoretic Implications}

The information concentration $\alpha_\mathcal{I}$ suggests that \emph{where decisions can be made}—i.e., where high operational information resides—is precisely at boundaries. This aligns with agency-field models where consciousness/agency emerges at interfaces of organized complexity.

\subsection{Mathematical Challenges}

Proving G1 and G2 rigorously is non-trivial. The 1D case may admit direct calculus-of-variations arguments. Higher dimensions require $\Gamma$-convergence machinery and careful treatment of exponential tails. We anticipate weaker liminf bounds may suffice for initial validation.

\subsection{Limitations}

\begin{itemize}
    \item \textbf{Pulled-front restriction:} The conjecture does not address pushed fronts (nonlinear speed selection). Extension to pushed regimes is an open question.
    \item \textbf{Dimensionality:} Higher dimensions ($d>3$) may exhibit different scaling; preliminary arguments suggest $d=4$ is a marginal case.
    \item \textbf{Non-scalar fields:} Vector or tensor fields require generalization of the partition concept.
\end{itemize}

\subsection{Relation to Existing Literature}

The conjecture synthesizes:
\begin{itemize}
    \item \textbf{Pulled fronts:} van Saarloos~\cite{vanSaarloos2003} established $c_\star=2\sqrt{Dr}$ selection.
    \item \textbf{$\Gamma$-convergence:} Modica--Mortola~\cite{ModicaMortola1977} showed perimeter-energy concentration.
    \item \textbf{Metriplectic structure:} Morrison~\cite{Morrison1986} formalized $J/M$ splitting.
    \item \textbf{Hierarchical scaling:} Kadanoff's renormalization ideas~\cite{Kadanoff1966} inform the scale-gap ratios.
\end{itemize}
Novel element: unifying these via the tachyonic-tail-to-hierarchy mechanism with testable operational predictions.

% =========================================================
\section{Conclusions}
\label{sec:conclusions}

\subsection{Summary}

We have formalized the \emph{Lietz Infinity Resolution Conjecture}: metriplectic scalar-field systems with tachyonic origin and pulled fronts must, to maintain finite excess energy on unbounded domains, organize into a finite-depth hierarchical partition with logarithmic depth, scale-gap separation, and boundary energy/information concentration. This provides a resolution of infinity via spontaneous emergence of structure.

\subsection{Next Steps}

\textbf{Immediate:}
\begin{enumerate}
    \item Analytical work on G1 (1D lower bound) and G2 ($\Gamma$-convergence sketch).
    \item Numerical campaign: implement preregistered experiment suite for G3--G5.
    \item Gate assessment: compile pass/fail report.
\end{enumerate}

\textbf{Medium-term:}
\begin{enumerate}
    \item Robustness testing (G6) across potentials, BCs, and meshes.
    \item Cross-code validation (G7) via independent implementation.
    \item Documentation compliance (G8).
\end{enumerate}

\textbf{Long-term:}
\begin{enumerate}
    \item Extension to vector fields (Yang--Mills, gravity).
    \item Application to VDM cosmogenesis: predict observable structure statistics.
    \item Connection to quantum field renormalization: does hierarchy emergence relate to UV/IR decoupling?
\end{enumerate}

\subsection{Promotion to Axiom A8}

Upon PASS (all G1--G5 + one of G6--G7 + G8), the conjecture will be elevated to \textbf{Axiom A8} in VDM Canon with the exact statement from Conjecture~\ref{conj:main}. Until then, it remains a \textbf{Tier-8 Axiom Candidate}.

% =========================================================
\section{Runtime and Scaling Disclosure}
\label{sec:runtime}

Numerical experiments will disclose runtime statistics to ensure reproducibility and assess computational feasibility.

\begin{table}[t]
  \centering
  \caption{Runtime and scaling disclosure (to be populated post-experiment).}
  \label{tab:runtime}
  \begin{tabular}{@{}lrrrr@{}}
  \toprule
  Metric & P50 & P95 & P99 & Notes \\
  \midrule
  Step time (ms) & --- & --- & --- & Per RK4 substep \\
  Active-site fraction & --- & --- & --- & Sites with $|\nabla\phi|>\epsilon$ \\
  Slope $\beta$ (log--log) & --- & --- & --- & CI: [\,\,] \\
  \bottomrule
  \end{tabular}
\end{table}

Preliminary estimates: 2D runs on $512^2$ grids take $\sim 10^3$ steps at $\sim 10$ ms/step (GPU); 3D on $128^3$ grids scale to $\sim 10^4$ steps at $\sim 100$ ms/step. Active-site fraction expected $\sim 0.1$--$0.3$ (boundary-dominated).

% =========================================================
\section*{Acknowledgments}

The author thanks the Variational Dynamics Model community for feedback on draft versions, and acknowledges computational resources provided by Neuroca, Inc. This work builds on foundational contributions from W.\ van Saarloos (pulled fronts), P.\ Morrison (metriplectic structure), and L.\ Modica \& S.\ Mortola ($\Gamma$-convergence theory).

% =========================================================
\bibliographystyle{unsrtnat}
\begin{thebibliography}{9}

\bibitem{vanSaarloos2003}
van Saarloos, W. (2003).
\newblock Front propagation into unstable states.
\newblock \emph{Physics Reports}, 386(2-6), 29--222.
\newblock \doi{10.1016/j.physrep.2003.08.001}

\bibitem{ModicaMortola1977}
Modica, L., \& Mortola, S. (1977).
\newblock Un esempio di $\Gamma$-convergenza.
\newblock \emph{Bollettino dell'Unione Matematica Italiana}, 14(1), 285--299.

\bibitem{Morrison1986}
Morrison, P. J. (1986).
\newblock A paradigm for joined Hamiltonian and dissipative systems.
\newblock \emph{Physica D: Nonlinear Phenomena}, 18(1-3), 410--419.
\newblock \doi{10.1016/0167-2789(86)90209-5}

\bibitem{Kadanoff1966}
Kadanoff, L. P. (1966).
\newblock Scaling laws for Ising models near $T_c$.
\newblock \emph{Physics Physique Fizika}, 2(6), 263--272.
\newblock \doi{10.1103/PhysicsPhysiqueFizika.2.263}

\bibitem{VDM_Canon}
Lietz, J. K. (2025).
\newblock Variational Dynamics Model: Canon and Derivation.
\newblock Repository: \texttt{github.com/justinlietz93/Prometheus\_VDM}.

\bibitem{VDM_EQUATIONS}
Lietz, J. K. (2025).
\newblock VDM Canonical Equations.
\newblock \texttt{Derivation/EQUATIONS.md}, commit 09f871a.

\bibitem{VDM_AXIOMS}
Lietz, J. K. (2025).
\newblock VDM Axioms and Foundational Principles.
\newblock \texttt{Derivation/AXIOMS.md}.

\bibitem{VDM_PROPOSAL_A8}
Lietz, J. K. (2025).
\newblock T8--A8 Proposal: Lietz Infinity Resolution Conjecture (v1).
\newblock \texttt{Derivation/T8\_A8\_PROPOSAL\_Lietz\_Infinity\_Conjecture\_v1.md}.

\bibitem{VDM_ROADMAP}
Lietz, J. K. (2025).
\newblock VDM Roadmap and Open Questions.
\newblock \texttt{Derivation/ROADMAP.md} and \texttt{Derivation/OPEN\_QUESTIONS.md}.

\end{thebibliography}

\end{document}
